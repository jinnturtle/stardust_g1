\documentclass[a4paper,10pt]{article}
\usepackage{main}
\usepackage[a4paper, textwidth=170mm, textheight=240mm]{geometry}

\begin{document}
\newlength{\tabcolsepDefault}
\setlength{\tabcolsepDefault}{\tabcolsep}

%-------------------------------------------------------------------------------

\definecolor{green1}{HTML}{A0F0C0}
\definecolor{tableColor1}{HTML}{FFFFFF}
\definecolor{tableColor2}{HTML}{F0E0D0}

%-------------------------------------------------------------------------------

\newcommand{\textbi}[1]{\textbf{\textit{#1}}}

\newcommand{\projName}{Stardust G1}

%-------------------------------------------------------------------------------

\title{Stardust G1 Game Design Document}
\author{Jinnturtle}

\maketitle
\tableofcontents

\TODO{change colors to bright on dark, for a more in-space feel, maybe use a
picture as background}

\TODO{incorporate svg diagrams and png thumbnails into doc}

\newpage
\part{Gameplay}

This part covers ideas and concepts of gameplay while avoiding mention of
technical requirements or constraints, technical concepts are discussed in
the part of this document dedicated to those subjects.

\section{Synopsis}
\begin{multicols}{2}
Project \projName{} game idea is to produce a space opera game, main concepts
include: flying a modular spaceship (chose from smallest boats to a gigantic
flying city), influencing the economy via providing services on small (e.g.
making cargo runs, shooting pirates) or large scale services/production/trade
via owning/operating some facilities/personell on a station or even growing into
a comporation with holdings in many star systems, exploring a vast interstellar
sandbox world.

Aspects in order of priority:
\begin{enumerate}
    \item the spacehsip - a home away from home, the office, the job
    \item production, crafting
    \item extracting, procuring raw materials; processing
    \item trading
    \item exploration, combat
\end{enumerate}

\TODO{Create a section for game concepts, and write a summary for each of the
aforementioned above.}
\end{multicols}

\section{Flavour}
\begin{multicols}{2}
Initial sources of inspiration EVE Online, X series, Alien movies, Alien
RPG by Free League, Coriolis RPG by Free League, Cyberpunk genre (books,
games, movies).

Some flavour text: \textit{``The hulking metal mass of the ship drifts ever so
slowly towards the docking clamps. Dull dings and clangs echo throughout the
hull as the docking mechanism locks the metallic star-farer in place and gently
starts pulling it towards the loading bays. Some time later rugged crews of men
and machinery start unloading the vital supplies which will go a long way
towards ensuring the colony's existence for the coming months. Ship crew float
out one by one into the station, hopeful with their own plans for the time
portside, maybe new experiences to break the routine, a visit to the colony,
some extra pay from a personal stash of pleasant bits and bobs rarely found out
here. The captain already thinking about avenues for the ship's next jobs while
keeping an eye on the unloading operation, can't leave the system with empty
holds after all, not if they want to keep the ship that is, if all goes well
they will even call her their own one day.''}

\TODO{flesh out this flavour/inspiration section a bit more with look/feel and
gameplay ideas}
\end{multicols}

\section{Economy}
\begin{multicols}{2}
The general idea of the game economy is that everything can be produced my the
players, and nothing produced is permanent i.e. value is absorbed back into the
economy and uncontrolled devaluation of currency (e.g. credits) doesn't happen
due to object permanence.
\end{multicols}

\section{Crafting}
\begin{multicols}{2}
Medium to large scale crafting for most people is performed via ordering a
job from a company, options and location of job execution depends on equipment
and personnell available at chosen facility(ies). Only the very affluent can
afford owning entire assembly lines and personnel.

The player should be able to make everything they can see/use (or as close to
everything as plausible) given enough effort.

Everything made should not be permanent: fuel/ammo/power cells are consumed by
ships, stations, colonies; food is consumed by crew, colony population, etc;
ships and installations get destroyed in conflict, most need maintenance; most
blueprints are of limited use i.e. serve as licence to produce a limited amount
of product;

A player should be able to extract or otherwise procure materials and produce
everything starting at any stage of the manufacturing process, but everything
will eventually get used up, wear out, or get destroyed in conflict.

Players should be able to automate production and supply lines to some extent,
as the idea is not to be forced into a mandatory click-fest once one's industry
empire grows a bit.

\subsection{Large Assembly (Ships/Stations, Station/Colony Facilities)}
Large structures should usually consist of a hull/frame as base and modules
that can be mounted on it.

A planetary settlement/colony or a larger space station will most likely consist
of several facilities that each have a frame/base and modules.
\end{multicols}



\section{Navigation}
\begin{multicols}{2}

\subsection{Space}
The interface should allow for direct keyboard control e.g. W/S - up/down, A/D
- strafe, Q/E - roll. Most manoeuvres should be comfortable to execute by mouse
via a commands menu (e.g. ``orbit target keeping 2km range''), as roll is
important for weapon fire manual keyboard control should be accessible, but
setting a behaviour e.g. (keep port/starboard/etc towards target) should be
enough for most use cases.

\subsection{Interstellar}
Most larger ships like freighters and military vessels are fitted with a warp
drive, when engaged this drive warps space-time around the ship effectively
creating gravity forces that propel the ship through space. Due to the nature
of the gravitational fields the drive creates, the performance of the drive
drops greatly near other significant gravitation fields (e.g. planets). It is
also dangerous to engage the warp drive even at low power closer than a few
thousand kilometres to heavier (a few kilograms or more) objects - usually the
warp fields dissipate instantly without much consequence, but there is
significant risk of catastrophic consequences to all objects involved
especially at the point of origin. The usual warp drive of this age propels
ships through interstellar space at 10 to 20 thousand g of acceleration when at
full power.

\note{For reference, at 10k g acceleration a vessel can reach 20ly distance in
just over two weeks, accounting for an acceleration and a deceleration phase.}

Orbital ships, persons, and smaller cargo shipments routinely make use of
so-called warp ferries - large ships with powerful warp drives that can create
a warp bubble large enough to safely transport nearby ships and large amounts
of cargo and people on board. The ferries are usually only available on
high-volume traffic lanes, which creates a big market for smaller freighters to
service the large amount of small colonies and fringe worlds. If your
destination is away from the ferry lanes, your best bet is to be on one of
those freighters, there may not be a stable schedule for flights but most ships
have a few cryo-pods available if you're willing to pay for their trouble and
the extra life-support.

\subsection{Stations and Colonies}
Most if not all facilities and options for in-station or on-planet operations
should be available remotely from other outposts or from space when close
enough.

When in/near station or on planet the various facilities can be accessed via
UI, most likely in the form of menus and/or buttons.

It is possible to dock ones ship with space and planetary installations if the
installation has facilities large enough to accommodate said ship. Some
operations require one to dock e.g. cargo or passenger transfer, neural
cloning, etc.

\subsection{Note on Large Settlements}
Navigating on large planets or in large multi-section space structures may
need additional thought (e.g. an interface to move between nodes/regions of
significance).

Hopefully this can be streamlined so additional handling would not be necessary
or would be minimal.

\TODO{Think about how this could work and if special handling is really needed,
I'm starting to lean towards a no. Graphical representation of largely populated
planets and multi-node space stations could be quite cool though.}
\end{multicols}


\section{Combat}
\begin{multicols}{2}
Main aspects to consider in combat is range and type of damage.

A weapon will deal certain kind(s) of damage depending on type of weapon chosen
and ammunition loaded. Weapon/ammo combinations will be most effective at a
certain range due to tracking/accuracy/ordnance stability concerns, so the armed
vessel in question will want to maintain this optimal range -- usually by being
faster than opposing craft.

\subsection{Damage Types and Damage Mitigation}
Weapons/ammunition will usually be most effective at dealing one or two types
of damage, as well as armor/shield defences can be calibrated to be most
effective at protecting from a limited range of damage types while being more
susceptible to other types of damage as a trade off.
\end{multicols}

Damage types:
\begin{itemize}
    \item kinetic (brute impact)
    \item heat
    \item static (electric charge / disruption, etc)
    \item radiation
\end{itemize}


\section{TODO}
\begin{itemize}
    \item Cover the manufacturing process from raw resource to module/ship,
    give a few examples.
    \item What's the default/main POV?
    \item more details on navigation in space short vs long distances, stars
    \item more on ship modularity and modules themselves
    \item weapon modules: types, concerns, etc
    \item large structure modularity (planetary colonies probably similar idea)
\end{itemize}

\part{Technical Implementation - Prototype 1}

\section{Objectives}
This first prototype is intended to act as a poc to get a very rough feeling
for some of the primary game systems.

\begin{itemize}
    \item Steps needed and time and circumstance needed for various actions in
        industry.
    \begin{itemize}
        \item docking/leaving station
        \item managing inventory in station cargo holds, renting hold/hangar space
        \item locating potential mining sites, or buying info
        \item prospecting potential mining sites, or buying info
        \item mining with a ship
        \item transporting ore
        \item refining ore
        \item manufacturing simple items
        \item selling (ore, refined minerals, manufactured items, etc)
    \end{itemize}
\end{itemize}

\section{Requirements}

\begin{itemize}
    \item Colour text rendering
    \item Icon graphics rendering
    \item Menus - lists of named (or graphical) options
    \item Timers, ability to display time remaining
    \item Item tables (inventory lists, system object lists, etc)
    \item 2d star system map, contains asteroid belt(s) and station(s) to
        interact with, has zoom function
\end{itemize}

\end{document}
