\documentclass[a4paper,10pt]{article}
\usepackage{main}
\usepackage[a4paper, textwidth=170mm, textheight=240mm]{geometry}

\begin{document}
\newlength{\tabcolsepDefault}
\setlength{\tabcolsepDefault}{\tabcolsep}

%-------------------------------------------------------------------------------

\definecolor{green1}{HTML}{A0F0C0}
\definecolor{tableColor1}{HTML}{FFFFFF}
\definecolor{tableColor2}{HTML}{F0E0D0}

%-------------------------------------------------------------------------------

\newcommand{\textbi}[1]{\textbf{\textit{#1}}}

\newcommand{\projName}{Stardust G1}

%-------------------------------------------------------------------------------

\title{Stardust G1 Game Design Document}
\author{Jinnturtle}

\maketitle
\tableofcontents

\TODO{change colors to bright on dark, for a more in-space feel, maybe use a
picture as background}

\TODO{incorporate svg diagrams and png thumbnails into doc}

\part{Gameplay}

This part covers ideas and concepts of gameplay while avoiding mention of
technical requirements or constraints, technical concepts are discussed in
the part of this document dedicated to those subjects.

\section{Synopsis}
\begin{multicols}{2}
Project \projName{} game idea is to produce a space opera game, main concepts
include: flying a modular spaceship (chose from smallest boats to a gigantic
flying city), influencing the economy via providing production/trade/services,
exploring a vast interstellar sandbox world.

Aspects in order of priority:
\begin{enumerate}
    \item production, crafting
    \item extracting, procuring raw materials; processing
    \item trading;
    \item exploration; combat
\end{enumerate}

\TODO{Create a section for game concepts, and write a summary for each of the
aforementioned above.}
\end{multicols}

\section{Flavour}
\begin{multicols}{2}
A lot can be drawn from EVE Online, X series

\TODO{flesh out this flavour/inspiration section a bit more with look/feel and
gameplay ideas}
\end{multicols}

\section{Economy}
\begin{multicols}{2}
The general idea of the game economy is that everything can be produced my the
players, and nothing produced is permanent i.e. value is absorbed back into the
economy and uncontrolled devaluation of currency (e.g. credits) doesn't happen
due to object permanence.
\end{multicols}

\section{Crafting}
\begin{multicols}{2}
the player should be able to make everything he can see/use (or as close to
everything as plausible) given enough effort.

Everything made should not be permanent: fuel/ammo/power cells are consumed by
ships, stations, colonies; food is consumed by crew, colony population, etc;
ships and installations get destroyed in conflict, most need maintenance; most
blueprints are of limited use i.e. serve as licence to produce a limited amount
of product;

A player should be able to extract or otherwise procure materials and produce
everything starting at any stage of the manufacturing process, but everything
will eventually get used up, wear out, or get destroyed in conflict.

Players should be able to automate production and supply lines to some extent,
as the idea is not to be forced into a mandatory click-fest once one's industry
empire grows a bit.

\subsection{Large Assembly (Ships/Stations, Station/Colony Facilities)}
Large structures should usually consist of a hull/frame as base and modules
that can be mounted on it.

A planetary settlement/colony or a larger space station will most likely consist
of several facilities that each have a frame/base and modules.
\end{multicols}


\section{Combat}
\begin{multicols}{2}
Main aspects to consider in combat is range and type of damage.

A weapon will deal certain kind(s) of damage depending on type of weapon chosen
and ammunition loaded. Weapon/ammo combinations will be most effective at a
certain range due to tracking/accuracy/ordnance stability concerns, so the armed
vessel in question will want to maintain this optimal range -- usually by being
faster than opposing craft.

\subsection{Damage Types and Damage Mitigation}
Weapons/ammunition will usually be most effective at dealing one or two types
of damage, as well as armor/shield defences can be calibrated to be most
effective at protecting from a limited range of damage types while being more
susceptible to other types of damage as a trade off.
\end{multicols}

Damage types:
\begin{itemize}
    \item kinetic (brute impact)
    \item heat
    \item static (electric charge / disruption, etc)
    \item radiation
\end{itemize}


\section{Questions to Answer/Develop}
\begin{itemize}
    \item pov
    \item navigation inside/on structures/planets/colonies, etc
\end{itemize}

\part{Technical Implementation}

\end{document}
